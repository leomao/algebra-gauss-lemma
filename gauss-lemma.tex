\documentclass[a4paper]{article}

%%%%%%%%%%%%%%%%%%page size%%%%%%%%%%%%%%%%%%
% \paperwidth=65cm
% \paperheight=160cm

%%%%%%%%%%%%%%%%%%%Package%%%%%%%%%%%%%%%%%%%
\usepackage[margin=3cm]{geometry}
\usepackage{mathtools,amsthm,amssymb}
\usepackage{centernot}
\usepackage{yhmath}
\usepackage{graphicx}
\usepackage{fontspec}
\usepackage{titlesec}
\usepackage{titling}
\usepackage{fancyhdr}
\usepackage{tabularx}
\usepackage[square, comma, numbers, super, sort&compress]{natbib}
\usepackage[usenames, dvipsnames]{color}
\usepackage[shortlabels, inline]{enumitem}
\usepackage{xpatch}
\usepackage{imakeidx}
\usepackage{bm}
\usepackage{hvindex}
%%% fix bugs in hvindex .................
\def\IndexXXii#1!#2@#3@#4\IndexNIL{%
  \ifx\relax#3\relax            %               no @ in last arg
    \def\hvTemp{#2}%
    \ifx\hvTemp\hvEncap\index{#1!{#2}}#2\else
      \ifx\hvIDXfont\hvIDXfontDefault\index{#1!{#2}}#2% <--- Here a % and #1 were missing
      \else\index{#1!#2@\hvIDXfont{#2}}\hvIDXfont{#2}\fi\fi%
  \else\index{#1!\protect#2@#3}#3\fi}

\def\IndexXXiii#1!#2!#3@#4@#5\IndexNIL{%
  \ifx\relax#4\relax 		% 		no @ in last arg
    \def\hvTemp{#3}%
    \ifx\hvTemp\hvEncap\index{#1!#2!{#3}}#3\else
      \ifx\hvIDXfont\hvIDXfontDefault\index{#1!#2!{#3}}#3
      \else\index{#1!#2!#3@\hvIDXfont{#3}}\hvIDXfont{#3}\fi\fi%
  \else\index{#1!#2!\protect#3@#4}#4\fi}
\usepackage[unicode, pdfborder={0 0 0}, bookmarksdepth=-1]{hyperref}
\hypersetup{
    colorlinks,
    linkcolor=red,
  }
\makeindex[columns=2, options= -s index_style.ist]

%\usepackage{tabto}     
%\usepackage{soul}      
%\usepackage{ulem}      
%\usepackage{wrapfig}   
%\usepackage{floatflt}  
\usepackage{float}     
\usepackage{caption}   
\usepackage{subcaption}
%\usepackage{setspace}  
\usepackage{mdframed}  
%\usepackage{multicol}  
%\usepackage[abbreviations]{siunitx}
%\usepackage{dsfont}   

%%%%%%%%%%%%%%%%%%%TikZ%%%%%%%%%%%%%%%%%%%%%%
\usepackage{tikz}
\usepackage{tikz-cd}
%\usepackage{circuitikz}
\usetikzlibrary{calc}
\usetikzlibrary{arrows}
\usetikzlibrary{shapes}
\usetikzlibrary{positioning}

\tikzstyle{every picture}+=[remember picture]

%%%%%%%%%%%%%%中文 Environment%%%%%%%%%%%%%%%
\usepackage[CheckSingle, CJKmath]{xeCJK}  % xelatex 中文
\usepackage{CJKulem}	% 中文字裝飾
\setCJKmainfont[BoldFont=Noto Sans CJK TC Bold]{Noto Serif CJK TC}
\setCJKsansfont[BoldFont=Noto Sans CJK TC Bold]{Noto Serif CJK TC}
\setCJKmonofont[BoldFont=Noto Sans CJK TC Bold]{Noto Serif CJK TC}

%%%%%%%%%%%%%%%%%font size%%%%%%%%%%%%%%%%%%%
\renewcommand{\baselinestretch}{1.1}
%\def\normalsize{\fontsize{10}{15}\selectfont}
%\def\large{\fontsize{12}{18}\selectfont}
%\def\Large{\fontsize{14}{21}\selectfont}
%\def\LARGE{\fontsize{16}{24}\selectfont}
%\def\huge{\fontsize{18}{27}\selectfont}
%\def\Huge{\fontsize{20}{30}\selectfont}

%%%%%%%%%%%%%%%Theme Input%%%%%%%%%%%%%%%%%%%
%\input{themes/chapter/neat}
%\input{themes/env/problist}

%%%%%%%%%%%titlesec settings%%%%%%%%%%%%%%%%%
%\titleformat{\chapter}{\bf\Huge}
            %{\arabic{section}}{0em}{}
%\titleformat{\section}{\centering\Large}
            %{\arabic{section}}{0em}{}
%\titleformat{\subsection}{\large}
            %{\arabic{subsection}}{0em}{}
%\titleformat{\subsubsection}{\bf\normalsize}
            %{\arabic{subsubsection}}{0em}{}
%\titleformat{command}[shape]{format}{label}
            %{gutter}{before}[after]

%%%%%%%%%%%%variable settings%%%%%%%%%%%%%%%%
%\numberwithin{equation}{section}
%\setcounter{secnumdepth}{4}
%\setcounter{tocdepth}{1}
%\setcounter{section}{0}
%\graphicspath{{images/}}

%%%%%%%%%%%%%%%page settings%%%%%%%%%%%%%%%%%
\newcolumntype{C}[1]{>{\centering\arraybackslash}p{#1}}
\setlength{\headheight}{15pt}  % with titling
\setlength{\droptitle}{-2.5cm}
%\posttitle{\par\end{center}}  % distance between title and content
\parindent=0pt % indent size
\parskip=1ex    % line space
%\pagestyle{empty}  % empty: no page number
%\pagestyle{fancy}  % fancy: fancyhdr

% use with fancygdr
%\lhead{\leftmark}
%\chead{}
%\rhead{}
%\lfoot{}
%\cfoot{}
%\rfoot{\thepage}
%\renewcommand{\headrulewidth}{0.4pt}
%\renewcommand{\footrulewidth}{0.4pt}

%\fancypagestyle{firststyle}
%{
  %\fancyhf{}
  %\fancyfoot[C]{\footnotesize Page \thepage\ of \pageref{LastPage}}
  %\renewcommand{\headrule}{\rule{\textwidth}{\headrulewidth}}
%}

\setlist{itemsep=0em, topsep=0.2em}

%%%%%%%%%%%%%%%renew command%%%%%%%%%%%%%%%%%
% \renewcommand{\contentsname}{Table of Content}
% \renewcommand{\refname}{Reference}
\renewcommand{\abstractname}{\LARGE Abstract}

%%%%%%%%symbol and function settings%%%%%%%%%
% adjust \exists and \forall spacing
\let\existstemp\exists
\let\foralltemp\forall
\renewcommand*{\exists}{\existstemp\mkern4mu}
\renewcommand*{\forall}{\foralltemp\mkern4mu}

\DeclarePairedDelimiter{\abs}{\lvert}{\rvert}
\DeclarePairedDelimiter{\norm}{\lVert}{\rVert}
\DeclarePairedDelimiter{\inpd}{\langle}{\rangle} % inner product
\DeclarePairedDelimiter{\ceil}{\lceil}{\rceil}
\DeclarePairedDelimiter{\floor}{\lfloor}{\rfloor}
\DeclareMathOperator{\adj}{adj}
\DeclareMathOperator{\sech}{sech}
\DeclareMathOperator{\csch}{csch}
\DeclareMathOperator{\arcsec}{arcsec}
\DeclareMathOperator{\arccot}{arccot}
\DeclareMathOperator{\arccsc}{arccsc}
\DeclareMathOperator{\arccosh}{arccosh}
\DeclareMathOperator{\arcsinh}{arcsinh}
\DeclareMathOperator{\arctanh}{arctanh}
\DeclareMathOperator{\arcsech}{arcsech}
\DeclareMathOperator{\arccsch}{arccsch}
\DeclareMathOperator{\arccoth}{arccoth}
\newcommand{\np}[1]{\\[{#1}] \indent}
\newcommand{\tr}[1]{{#1}^\mathrm{t}}
%%%% Geometry Symbol %%%%
\newcommand{\degree}{^\circ}
\newcommand{\Arc}[1]{\wideparen{{#1}}}
\newcommand{\Line}[1]{\overleftrightarrow{{#1}}}
\newcommand{\Ray}[1]{\overrightarrow{{#1}}}
\newcommand{\Segment}[1]{\overline{{#1}}}
%%%% Math symbol %%%%
\newcommand{\defeq}{\vcentcolon=}
\newcommand{\Nb}{\mathbb{N}}
\newcommand{\Zb}{\mathbb{Z}}
\newcommand{\Qb}{\mathbb{Q}}
\newcommand{\Rb}{\mathbb{R}}
\newcommand{\Cb}{\mathbb{C}}
\newcommand{\Hb}{\mathbb{H}}
\newcommand{\Fb}{\mathbb{F}}
\newcommand{\Fbx}{\mathbb{F}^\times}
\newcommand{\Qbx}{\mathbb{Q}^\times}
\newcommand{\Rbx}{\mathbb{R}^\times}
\newcommand{\Cbx}{\mathbb{C}^\times}
\newcommand{\Hbx}{\mathbb{H}^\times}
\newcommand{\Gc}{\mathcal{G}}
\newcommand{\Fc}{\mathcal{F}}
\newcommand{\Cc}{\mathcal{C}}
\newcommand{\Dc}{\mathcal{D}}
\newcommand{\sT}{{\sf T}}
\newcommand{\sI}{{\sf I}}
\newcommand{\Id}{{\rm Id}}

\newcommand{\Mf}{\mathfrak{M}}
\newcommand{\Grf}{\mathfrak{Gr}}

\newcommand{\bigOp}{\raisebox{0.3ex}{$\bigoplus$}}
\newcommand{\bigOt}{\raisebox{0.3ex}{$\bigotimes$}}
\newcommand{\Op}{\oplus}
\newcommand{\Ot}{\otimes}

\DeclareMathOperator{\Sym}{Sym}
\DeclareMathOperator{\Alt}{Alt}
\DeclareMathOperator{\diag}{diag}
\DeclareMathOperator{\sgn}{sgn}
\DeclareMathOperator{\lcm}{lcm}
\DeclareMathOperator{\Image}{Im}
\DeclareMathOperator{\Char}{char}
\DeclareMathOperator{\Fix}{Fix}
\DeclareMathOperator{\Inn}{Inn}
\DeclareMathOperator{\Aut}{Aut}
\DeclareMathOperator{\Gal}{Gal}
\DeclareMathOperator{\Isom}{Isom}
\DeclareMathOperator{\Tor}{Tor}
\DeclareMathOperator{\Exp}{Exp}
\DeclareMathOperator{\Syl}{Syl}

% multilinear
\DeclareMathOperator{\Hom}{Hom}
\DeclareMathOperator{\Lrad}{lrad}
\DeclareMathOperator{\Rrad}{rrad}
\DeclareMathOperator{\rank}{rank}
\DeclareMathOperator{\trace}{Tr}

% extensions
\DeclareMathOperator{\Stab}{Stab}
\DeclareMathOperator{\Der}{Der}
\DeclareMathOperator{\PDer}{PDer}
\DeclareMathOperator{\Ext}{Ext}

\DeclareMathOperator{\trdeg}{\mathrm{tr} \deg}
%
\newcommand{\Resf}[1]{\big|_{#1}}

\newcommand{\ob}{\overline}
\DeclareMathOperator{\ord}{ord}
\DeclarePairedDelimiter{\gen}{\langle}{\rangle} % generator
%\newcommand*\quot[2]{{^{\textstyle #1}\Big/_{\textstyle #2}}}
\newcommand*\quot[2]{{#1}/{#2}}
\newcommand\bij{\lhook\joinrel\twoheadrightarrow}
\newcommand\toone{\hookrightarrow}
\newcommand\onto{\twoheadrightarrow}
\newcommand\acts{\curvearrowright}
\newcommand\revacts{\curvearrowleft}
\newcommand\isoto{\xrightarrow{\sim}}
\newcommand\deffunc[5]{\ensuremath{
  \arraycolsep=1pt
  \begin{array}{rccc}
    #1: & #2 & \to & #3 \\
       & #4 & \mapsto & #5
  \end{array}
}}

% just to make sure it exists
\providecommand\given{}
% can be useful to refer to this outside \Set
\newcommand*\SetSymbol[1][]{%
  \nonscript\:#1\vert
  \allowbreak
  \nonscript\:
\mathopen{}}
\DeclarePairedDelimiterX\Set[1]\{\}{%
  \renewcommand\given{\SetSymbol[\delimsize]}
  \,#1\,
}

\DeclarePairedDelimiterX\Gen[1]{\langle}{\rangle}{%
  \renewcommand\given{\SetSymbol[\delimsize]}
  \,#1\,
}

% cycle group \cycle{1,2,3} => (1 2 3)
\ExplSyntaxOn
\NewDocumentCommand{\cycle}{ O{\;} m }
 {
  (
  \alec_cycle:nn { #1 } { #2 }
  )
 }

\seq_new:N \l_alec_cycle_seq
\cs_new_protected:Npn \alec_cycle:nn #1 #2
 {
  \seq_set_split:Nnn \l_alec_cycle_seq { , } { #2 }
  \seq_use:Nn \l_alec_cycle_seq { #1 }
 }
\ExplSyntaxOff

\newcommand\Div{\mathrel{\big|}}
\newcommand\nDiv{\mathrel{\not\big|}}
\newcommand\relmiddle[1]{\mathrel{}\middle#1\mathrel{}}
\newcommand{\RNum}[1]{\uppercase\expandafter{\romannumeral #1\relax}}

%%%%%%%%%%%%%%%%%%%%%%%%%%%%%%%%%%%%%%%%%%%%
%\renewcommand{\proofname}{\bf pf:}
\newtheoremstyle{mystyle}% custom style
  {6pt}{15pt}%      top and bottom margin
  {}%               content style
  {}%               indent
  {\bf}%            head style
  {.}%              after head
  {1em}%            distance between head and content
  {}%               Theorem head spec (can be left empty, meaning 'normal')

\theoremstyle{mystyle}
\newtheorem{theorem}{Theorem}
\newtheorem{formula}{Formula}
\newtheorem{lemma}{Lemma}
\newtheorem{remark}{Remark}
\newtheorem{observation}{Observation}
\newtheorem*{observation*}{Observation}
\newtheorem{definition}{Def}
\newtheorem{exercise}{Ex}
\newtheorem{example}{Eg}
\newtheorem{fact}{Fact}
\newtheorem{prop}{Prop}
\newtheorem{coro}{Coro}

%%%%%%%%%%%%%%Title information%%%%%%%%%%%%%%
\title{Gauss's Lemma}
\author{Yao-Wen Mao}
\date{\today}

\begin{document}
\maketitle
% \thispagestyle{empty}
% \thispagestyle{fancy}
% \tableofcontents
%%%%%%%%%%%%%include file here%%%%%%%%%%%%%%%
We recap some basic definitions and facts first:

\begin{definition}[field of fractions]
  Let $R$ be a integral domain. The {\bf fraction field} $F$ of $R$
  is \[ F = \Set*{\frac{a}{b} \given a, b \in R, b \ne 0}. \]
  Sometimes we also use $\text{Frac}(R)$ to denote the field of fractions.
\end{definition}

\begin{fact} \mbox{}
  \begin{itemize}
    \item PID implies UFD, UFD implies GCD.
    \item If $R$ is an integral domain, then $R[x]$ is also an integral domain.
      (check the leading coefficient)
    \item Let $R$ be a ring and $I \subsetneq R$ is an ideal, then
      $I$ is a prime ideal $\iff$ $\quot{R}{I}$ is an integral domain.
    \item Let $R$ be a ring. $p \in R$ is a prime $\iff$
      $\gen{p}_R$ is a prime ideal.
    \item A prime element is irreducible.
    \item In GCD domains, an irreducible element is also a prime.
  \end{itemize}
\end{fact}

\begin{definition}
  Let $R$ be a GCD domain and $f = a_nx^n + \dots + a_1x + a_0 \in R[x]$.
  We define the {\bf content} of $f$ as $c(f) = \gcd(a_n, \dots, a_0)$.
\end{definition}

\begin{definition}
  Let $R$ be a GCD domain and $f = a_nx^n + \dots + a_1x + a_0 \in R[x]$.
  We say that $f$ is {\bf primitive} if $c(f) = 1$.
\end{definition}

\begin{theorem}[Gauss's Lemma]
  Let $R$ be a UFD and $f, g \in R[x]$ are primitive. Then $fg$
  is also primitive.
  \begin{proof}
    Prove by contradiction. Assume $fg$ is not primitive, then
    $\exists d \in R$ is non-invertible such that such that $d \mid c(fg)$.
    Since $R$ is a UFD, there exists an irreducible element $p \in R$
    such that $p \mid d$. Note that $p$ is also a prime. Let $I = \gen{p}_R$.
    We know that $\quot{R}{I}$ is an integral domain $\leadsto$
    $(\quot{R}{I})[x]$ is also an integral domain.
    Consider the projection map $\tau: R[x] \to (\quot{R}{I})[x]$ (which is
    a ring homo.), we have $0 = \tau(fg) = \tau(f)\tau(g)$. Since
    $(\quot{R}{I})[x]$ is an integral domain, one of $\tau(f), \tau(g)$ must
    be $0$. This implies that $p$ divides at least one of $c(f)$ and $c(g)$,
    which is a contradiction.
  \end{proof}
\end{theorem}

\begin{remark}
  Let $R$ be a UFD. It is obvious that $c(fg) = c(f)c(g)$ for $f, g \in R[x]$.
\end{remark}

\begin{theorem}
  Let $R$ be a UFD with the fraction field $F$. If $f \in R[x]$
  is reducible in $F[x]$, then $f$ is also reducible in $R[x]$.

  \begin{proof}
    Let $f = gh$ is reducible in $F[x]$ ($\deg g, \deg h > 1$).
    Since the coefficients of $g, h$ are elements in $F$, we can find
    $a, b \in R$ s.t. $ag, bh \in R[x]$. Let $r = c(ag), s = c(bh)$, then
    $g' = \frac{a}{r}g, h' = \frac{b}{s}h$ are primitive in $R[x]$.
    Let $\alpha = \frac{rs}{ab}$, then we can write $f = \alpha g' h'$.
    Notice that $g'h'$ is primitive $\implies ab = 1 \implies \alpha \in R$.
    Then $f = (\alpha g')  h'$, hence $f$ is reducible in $R[x]$.
  \end{proof}
\end{theorem}

\begin{remark}
  Let $f \in R[x]$. If $f = f_1 f_2 \dotsm f_n$ where $f_i \in F[x],\, \forall i$.
  Then $\exists a_1, \dots, a_n \in F$ s.t. $g_i = a_if_i \in R[x]$ and
  $f = g_1 g_2 \dotsm g_n$.
\end{remark}

\begin{remark}
  Let $R$ be a UFD with the fraction field $F$. $f \in R[x]$
  is irreducible $\iff$ $f$ is irreducible in $F[x]$ and primitive in $R[x]$.
\end{remark}

\begin{coro}[Rational Root Theorem]
  Let $f = a_n x^n + \dots + a_1 x + a_0 \in \Zb[x]$.
  If $p/q$ is a root of $f$. Then $q \mid a_n, p \mid a_0$.
  (Note $\gcd(p, q) = 1$)
  \begin{proof}
    If $p/q$ is a root of $f$. Then $f = (x - p/q)g$ for some $g \in \Qb[x]$.
    By Gauss's lemma, $\exists a \in \Qb$ s.t. $g' = ag \in \Zb[x]$ and
    $f = (qx - p)g'$. Then obviously $q \mid a_n, p \mid a_0$.
  \end{proof}
\end{coro}

\begin{theorem}
  $R$ is a UFD $\iff$ $R[x]$ is a UFD.
  \begin{proof}
    ``$\Leftarrow$'' is obvious. \\
    ``$\Rightarrow$'': Let $R$ be a UFD and $F$ be the fraction field of $R$.
    Let $f \in R[x]$, note that $c(f) \in R$ has a unique factorization,
    we only need to check that $f$ has a unique factorization when $f$ is primitive.
    Also, we can assume $\deg f > 0$.

    Notice that $F$ is a field, so $F[x]$ is a PID $\leadsto$ UFD.
    We can write $f = f_1f_2\dotsm f_n$ as a factorization in $F[x]$,
    $f_i$ is irreducible in $F[x]$.
    Then $\exists a_1, \dots, a_n \in F$ s.t.
    $f = g_1 g_2 \dotsm g_n$ where $g_i = a_if_i \in R[x]$.
    Note that $f$ is primitive $\implies$ each $g_i$ is primitive.
    Also, $f_i$ is irreducible in $F[x]$ implies $g_i$ is also irreducible in $F[x]$.
    So each $g_i$ is irreducible in $R[x]$.  We can conclude that 
    $f = g_1 g_2 \dotsm g_n$ is a factorization in $R[x]$.

    Finally, we check the uniqueness of the factorization. Assume
    $f = p_1p_2 \dotsm p_n = q_1q_2 \dotsm q_m$ where $p_i, q_j \in R[x]$.
    Notice that any factorization in $R[x]$ is also a factorization in $F[x]$.
    So $n = m$ and we can assume that for each $i = 1,\dots, n$,
    $p_i = r_i q_i$ where $r_i \in F$. For each $r_i$, we can write
    $r_i = \frac{a_i}{b_i}$ where $a_i, b_i \in R$. Then $b_i p_i = a_i q_i$.
    Since $p_i, q_i$ are primitive, it is obvious that
    $c(b_i p_i) = b_i, c(a_i q_i) = a_i \implies a_i = u_ib_i$ for some unit $u_i \in R$.
    $\implies p_i = u_i q_i$. So the factorization is unique up to units.
  \end{proof}
\end{theorem}

\begin{theorem}[Gauss's Lemma for GCD domains]
  Let $R$ be a GCD and $f, g \in R[x]$. If $f, g$ are pritive, then $fg$
  is also primitive. More generally, $c(fg) = c(f)c(g)$.
  \begin{proof}
    Notice that for any $f, g \in R[x]$, we have
    $c(fg) = c(f)c(g)c(f'g')$ where $f' = \frac{1}{c(f)} f, g' = \frac{1}{c(g)} g$.
    Clearly $f', g'$ are primitive. So we only need to prove when
    $f, g$ are primitive.

    Now we prove by induction on the total number of nonzero terms of $f$ and $g$.
    If one of $f, g$ has at most one term, then $fg$ is primitive obviously.

    Assume both $f, g$ have at least two terms.
    Let $f = a_n x^n + \dots + a_1 x + a_0$ and
    $g = b_m x^m + \dots + b_1 x + b_0$.
    Assume $fg$ is not primitive, then $c(fg)$ is not invertible.
    In particular, $c(fg) \mid a_n b_m$. Then at least one of $\gcd(c(fg), a_n), \gcd(c(fg), b_m)$
    must be non-trivial. WLOG, assume $d = \gcd(c(fg), a_n)$ is non-trivial.
    Then we have $s \mid c(fg)$ and $s \mid c( a_nx^n g ) \implies
    s \mid c( (f - a_nx^n) g)$. Notice that $f - a_n x^n$ has terms fewer than
    $f$, by induction hypothesis, $c( (f - a_nx^n) g) = c(f - a_n x^n) c(g)
    = c(f - a_n x^n)$. Then $s \mid c(f - a_n x^n)$. But we also have $s \mid a_n$.
    This implies $s \mid c(f)$, which is a contradiction.
    $\implies fg$ is primitive.
  \end{proof}
\end{theorem}

%%%%%%%%%%%%%%%%%%%%%%%%%%%%%%%%%%%%%%%%%%%%%
% \bibliographystyle{plain}
% \bibliography{journal.bib}
% \begin{thebibliography}{99}
% \bibitem[1]{ex}\url{http://www.example.com/}
% \end{thebibliography}
\end{document}
